\documentclass{article}
\usepackage[letterpaper, margin=0.1in]{geometry}
\usepackage{multicol}
\usepackage{graphicx}
\usepackage{amsmath}

\begin{document}
\LARGE
\section{Pythagoean Identities}
\hrule
\begin{multicols}{3}
    \noindent
    $\sin ^2 \theta + \cos ^2 \theta = 1$  \\
    $\sec ^2 \theta = 1 + \tan ^2 \theta$  \\
    $\csc ^2 \theta = 1 + \cot ^2 \theta$
\end{multicols}

\section{Cofunction Identities}
\hrule
\begin{multicols}{3}
    \noindent
    $\sin\theta=\cos\left(\dfrac{\pi}{2}-\theta\right)\\\\
    \cos\theta=\sin\left(\dfrac{\pi}{2}-\theta\right)$ \\
    $\sec\theta=\csc\left(\dfrac{\pi}{2}-\theta\right)\\\\
    \csc\theta=\sec\left(\dfrac{\pi}{2}-\theta\right)$\\
    $\tan\theta=\cot\left(\dfrac{\pi}{2}-\theta\right)\\\\
    \cot\theta=\tan\left(\dfrac{\pi}{2}-\theta\right)$
\end{multicols}

\section{Even Odd Identities}
\hrule
\begin{multicols}{3}
    \noindent
    $\sin(-\theta) = -\sin\theta$ \\
    $\csc(-\theta) = -\csc\theta$ \\
    $\tan(-\theta) = -\tan\theta$ \\
    $\cot(-\theta) = -\cot\theta$ \\
    $\cos(-\theta) = \cos\theta$ \\
    $\sec(-\theta) = \sec\theta$ 
\end{multicols}

\section{Supplement Angle Identities}
\hrule
\begin{multicols}{3}
    \noindent
    $\sin(\pi - \theta) = \sin \theta$ \\
    $\csc(\pi - \theta) = \csc \theta$ \\
    $\sin(\pi + \theta) = -\sin \theta$ \\
    $\csc(\pi + \theta) = -\csc \theta$ \\
    $\cos(\pi - \theta) = -\cos \theta$ \\
    $\sec(\pi - \theta) = -\sec \theta$ \\
    $\cos(\pi + \theta) = -\cos \theta$ \\
    $\sec(\pi + \theta) = -\sec \theta$ \\
    $\tan(\pi - \theta) = -\tan \theta$ \\
    $\cot(\pi - \theta) = -\cot \theta$ \\
    $\tan(\pi + \theta) = \tan \theta$ \\
    $\cot(\pi + \theta) = \cot \theta$ 
\end{multicols}

\section{Addition and Subtraction Identities}
\hrule
\begin{multicols}{2}
    \noindent
    $\sin(A + B) = \sin A \cos B + \cos A \sin B$ \\
    $\cos(A + B) = \cos A \cos B - \sin A \sin B$ \\
    $\tan(A + B) = \dfrac{\tan A + \tan B}{1- \tan A \ tan B}$\\
\end{multicols}
\begin{multicols}{2}
    \noindent
    $\sin(A - B) = \sin A \cos B - \cos A \sin B$ \\
    $\cos(A - B) = \cos A \cos B + \sin A \sin B$ \\
    $\tan(A - B) = \dfrac{\tan A - \tan B}{1 + \tan A \ tan B}$\\
\end{multicols}

\section{Double-Angle Identities}
\hrule
\begin{multicols}{3}
    \noindent
    $\sin(2\theta) = 2\sin\theta\cos\theta$ \\
    $\cos(2\theta) = \cos^2\theta-\sin^2\theta\\
    \cos(2\theta) = 1-2\sin^2\theta \\
    \cos(2\theta) = 2\cos^2\theta - 1$ \\
    $\tan(2\theta) = \dfrac{2\tan\theta}{1-\tan^2\theta}$
\end{multicols}
\end{document}